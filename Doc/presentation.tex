\documentclass[aspectratio=169]{beamer}
\usepackage{hyperref}
\usepackage{listings}
% \usepackage{fontspec}
% \setmonofont{Consolas}

\definecolor{background}{RGB}{39, 40, 34}
\definecolor{string}{RGB}{230, 219, 116}
\definecolor{comment}{RGB}{117, 113, 94}
\definecolor{normal}{RGB}{248, 248, 242}
\definecolor{identifier}{RGB}{166, 226, 46}
\definecolor{keyword}{HTML}{F92672}
\definecolor{numbers}{HTML}{AE81FF}
\definecolor{types}{HTML}{66D9EF}
% \lstset{
%   language=C++,                			% choose the language of the code
%   numbers=left,                  		% where to put the line-numbers
%   stepnumber=1,                   		% the step between two line-numbers.        
%   numbersep=5pt,                  		% how far the line-numbers are from the code
%   numberstyle=\tiny\color{normal}\ttfamily,
%   showspaces=false,               		% show spaces adding particular underscores
%   showstringspaces=false,         		% underline spaces within strings
%   showtabs=false,                 		% show tabs within strings adding particular underscores
%   tabsize=4,                      		% sets default tabsize to 2 spaces
%   captionpos=b,                   		% sets the caption-position to bottom
%   breaklines=true,                		% sets automatic line breaking
%   breakatnormalspace=true,         		% sets if automatic breaks should only happen at normalspace
%   basicstyle=\ttfamily,
%   keywordstyle=\color{identifier}\ttfamily,
%   stringstyle=\color{string}\ttfamily,
%   commentstyle=\color{comment}\ttfamily,
%   morecomment=[l][\color{string}]{\#}
% }

\lstset{
    language=C,
    tabsize=4, % tab space width
    showstringspaces=false, % don't mark spaces in strings
    numbers=left, % display line numbers on the left
    basicstyle=\ttfamily,
    numberstyle=\color{comment}\ttfamily, % Line numbers
    commentstyle=\color{comment}\ttfamily, % comment color
    otherkeywords={>,<,-,!,=,~}, % Color operators too
    morekeywords={>,<,-,!,=,~},
    keywordstyle=\color{keyword}\ttfamily, % keyword color
    stringstyle=\color{string}\ttfamily, % string color
    morecomment=[l][\color{keyword}]{\#},
    emph={int,char,long,float,double,unsigned},
    emphstyle={\color{types}\ttfamily\textit},
    escapeinside={@!}{!@}
}

\newcommand{\code}{\texttt}
\newcommand{\fn}{\color{identifier}}

\usetheme{Warsaw}
\setbeamercolor{normal text}{fg=normal,bg=background}
\setbeamercolor{structure}{fg=normal}

\setbeamercolor{alerted text}{fg=red!85!black}

\setbeamercolor{item projected}{use=item,fg=background,bg=item.fg!35}

\setbeamercolor*{palette primary}{use=structure,fg=structure.fg}
\setbeamercolor*{palette secondary}{use=structure,fg=structure.fg!95!black}
\setbeamercolor*{palette tertiary}{use=structure,fg=structure.fg!90!black}
\setbeamercolor*{palette quaternary}{use=structure,fg=structure.fg!95!black,bg=black!80}

\setbeamercolor*{framesubtitle}{fg=normal}

\setbeamercolor*{block title}{parent=structure,bg=background}
\setbeamercolor*{block body}{fg=black,bg=background}
\setbeamercolor*{block title alerted}{parent=alerted text,bg=background}
\setbeamercolor*{block title example}{parent=example text,bg=background}

\title{How to Use a Debugger}
\subtitle{Focusing on GDB}
\author{Richard Morrill}
\institute{Fordham University CS Society}
\logo{\includegraphics[width=2cm]{Images/css_logo_color.png}}
\date{Thursday, Novemeber 7th 2019}

\section{}
\begin{document}
\begin{frame}
\titlepage
\end{frame}

\section{Background Info}
\begin{frame}
    \frametitle{Intro}

    \begin{itemize}
        \item WTF is a debugger anyways?
        \item Why would I want one?
        \item How the heck do I set all this crap up?
    \end{itemize}
\end{frame}

\begin{frame}
    \frametitle{Setup}
    \framesubtitle{Let's get all the tech support out of the way.}
    \begin{itemize}
        \item Open Instructions Doc: \url{http://bit.ly/GDBSetup}
        \item Write / Compile a Simple C/C++ Program
        \item Run It, Make Sure it Works
        \item Run \code{\$ gdb <your executable name>}
        \item Run \code{(gdb)\footnote[frame]{This means you run the command inside GDB} run}
    \end{itemize}

\end{frame}

\begin{frame}[fragile]
    \frametitle{The Problem a Debugger Solves}
    % do_something
    \begin{lstlisting}
int @!\fn{doSomething}!@(int* arr, int len) {
    for (int i = 0; i < len; ++i) {
        if (arr[i] < 4) {
            arr[i] = @!\fn{complexFunction}!@(arr[i]);
        }
    }
    for(int i = 0; i < len; ++i) {
        if (@!\fn{simpleFunction}!@(arr[i])) {
            return arr[i];
        } else {
            arr[i] = @!\fn{doSomething}!@(arr, len);
        }
    }
}
    \end{lstlisting}

\end{frame}
\begin{frame}[fragile]
    \frametitle{The Problem a Debugger Solves, Cont.}
    \begin{itemize}
        \item I want to know where the program is crashing
        \item So I do this:
    \end{itemize}
    \begin{lstlisting}[numbers=none]
#include <stdio.h>
    \end{lstlisting}
    \begin{itemize}
        \pause
        \item And this\dots
    \end{itemize}
    \begin{lstlisting}[numbers=none]
@!\fn{printf}!@("some random variable: %d", var);
    \end{lstlisting}
    \begin{itemize}
        \pause
        \item My code is full of print statements
        \item My terminal is flooded with info
    \end{itemize}


\end{frame}
\begin{frame}
    \frametitle{What I Want to Do}

    \begin{itemize}
        \item Stop my program anywhere I want
        \item See what all the variables are
        \item See how they got that way
        \item If it crashes, see what was going on when it crashed
        \item Not need to delete all those stupid print statements afterward
    \end{itemize}

\end{frame}
\begin{frame}
    \frametitle{What a Debugger Can Do for Me}

    \begin{itemize}
        \item Stop my program: breakpoints, watchpoints, on exceptions
        \item See values of all variables (and registers!)
        \item Follow \underline{call stack} back up\footnote[frame]{Underlined terms are things not everybody will know, please ask me to explain them.}
        \item Play ``what if?''
    \end{itemize}

\end{frame}
\section{Using GDB}
\begin{frame}
    \frametitle{Essential Commands}
    \framesubtitle{I forget them constantly and so will you}
    \begin{itemize}
        \item Open this: \url{http://bit.ly/GDBCheatSheet}
        \item Essential Commands:
        \begin{itemize}
            \item \code{\$ gdb EXECUTABLE} -- start debugging
            \item \code{(gdb) run ARGS\dots} -- run your program inside debugger
            \item \code{(gdb) break [FUNCTION|FILE:LINE]} -- set \underline{breakpoint}
            \item \code{(gdb) step} -- \underline{single step}
            \item \code{(gdb) print EXPRESSION} -- print value
            \item \code{(gdb) kill} -- start over
            \item \code{(gdb) quit} -- exit
            \item \code{(gdb) continue} -- start running again
        \end{itemize}
    \end{itemize}
    

\end{frame}

\begin{frame}
    \frametitle{Time for Some Real Debugging}

    \begin{itemize}
        \item Navigate to the \code{Basic\_Usage} folder
        \item Compile with \code{\$~gcc~hasfunctions.c~-lm~-o~hasfunctions}\footnote[frame]{Anybody need me to explain what the arguments are doing?}
        \item Run it with 2 int \underline{arguments}\footnote[frame]{Anybody not know what that means?}
        \pause
        \item 90\% chance you'll get a segfault
        \item Run it in gdb
    \end{itemize}

\end{frame}
\end{document}