\documentclass[aspectratio=169]{beamer}
\usepackage{hyperref}
\usepackage{listings}

% \usepackage{fontspec}
% \setmonofont{Consolas}

\definecolor{background}{RGB}{39, 40, 34}
\definecolor{string}{RGB}{230, 219, 116}
\definecolor{comment}{RGB}{117, 113, 94}
\definecolor{normal}{RGB}{248, 248, 242}
\definecolor{identifier}{RGB}{166, 226, 46}
\definecolor{keyword}{HTML}{F92672}
\definecolor{numbers}{HTML}{AE81FF}
\definecolor{types}{HTML}{66D9EF}


\lstset{
    language=C,
    tabsize=4, % tab space width
    showstringspaces=false, % don't mark spaces in strings
    numbers=left, % display line numbers on the left
    basicstyle=\ttfamily,
    numberstyle=\color{comment}\ttfamily, % Line numbers
    commentstyle=\color{comment}\ttfamily, % comment color
    otherkeywords={>,<,-,!,=,~}, % Color operators too
    morekeywords={>,<,-,!,=,~},
    keywordstyle=\color{keyword}\ttfamily, % keyword color
    stringstyle=\color{string}\ttfamily, % string color
    morecomment=[l][\color{keyword}]{\#},
    emph={int,char,long,float,double,unsigned,namespace,typename},
    emphstyle={\color{types}\ttfamily\textit},
    escapeinside={@!}{!@},
    captionpos=b
}
% \lstdefinelanguage
%    [x64]{Assembler}     % add a "x64" dialect of Assembler
%    [x86masm]{Assembler} % based on the "x86masm" dialect
%    % with these extra keywords:
%    {morekeywords={CDQE,CQO,CMPSQ,CMPXCHG16B,JRCXZ,LODSQ,MOVSXD, %
%                   POPFQ,PUSHFQ,SCASQ,STOSQ,IRETQ,RDTSCP,SWAPGS, %
%                   rax,rdx,rcx,rbx,rsi,rdi,rsp,rbp, %
%                   r8,r8d,r8w,r8b,r9,r9d,r9w,r9b, %
%                   r10,r10d,r10w,r10b,r11,r11d,r11w,r11b, %
%                   r12,r12d,r12w,r12b,r13,r13d,r13w,r13b, %
%                   r14,r14d,r14w,r14b,r15,r15d,r15w,r15b}} % etc
\newcommand{\code}{\texttt}
\newcommand{\fn}{\color{identifier}}

\usetheme{Warsaw}

\addtobeamertemplate{footnote}{\vspace{-6pt}\advance\hsize-0.5cm}{\vspace{6pt}}
\makeatletter
% Alternative A: footnote rule
\renewcommand*{\footnoterule}{\kern -3pt \hrule \@width 2in \kern 8.6pt}
% Alternative B: no footnote rule
% \renewcommand*{\footnoterule}{\kern 6pt}
\makeatother

\setbeamercolor{normal text}{fg=normal,bg=background}
\setbeamercolor{structure}{fg=normal}

\setbeamercolor{alerted text}{fg=red!85!black}

\setbeamercolor{item projected}{use=item,fg=background,bg=item.fg!35}

\setbeamercolor*{palette primary}{use=structure,fg=structure.fg}
\setbeamercolor*{palette secondary}{use=structure,fg=structure.fg!95!black}
\setbeamercolor*{palette tertiary}{use=structure,fg=structure.fg!90!black}
\setbeamercolor*{palette quaternary}{use=structure,fg=structure.fg!95!black,bg=black!80}

\setbeamercolor*{framesubtitle}{fg=normal}

\setbeamercolor*{block title}{parent=structure,bg=background}
\setbeamercolor*{block body}{fg=black,bg=background}
\setbeamercolor*{block title alerted}{parent=alerted text,bg=background}
\setbeamercolor*{block title example}{parent=example text,bg=background}

\title{How to Use a Debugger}
\subtitle{Focusing on GDB}
\author{Richard Morrill}
\institute{Fordham University CS Society}
\logo{\includegraphics[width=2cm]{Images/css_logo_color.png}}
\date{Thursday, Novemeber 7th 2019}

\section{}
\begin{document}
\begin{frame}
\titlepage
\end{frame}

\section{Background Info}
\begin{frame}
    \frametitle{Intro}

    \begin{itemize}
        \item WTF is a debugger anyways?
        \item Why would I want one?
        \item How the heck do I set all this crap up?
    \end{itemize}
\end{frame}

\begin{frame}
    \frametitle{Setup}
    \framesubtitle{Let's get all the tech support out of the way.}
    \begin{itemize}
        \item Open Instructions Doc: \url{http://bit.ly/GDBSetup}
        \item Write / Compile a Simple C/C++ Program
        \item Run It, Make Sure it Works
        \item Run \code{\$ gdb <your executable name>}
        \item Run \code{(gdb)\footnote[frame]{This means you run the command inside GDB} run}
    \end{itemize}

\end{frame}

\begin{frame}[fragile]
    \frametitle{The Problem a Debugger Solves}
    % do_something
    \begin{lstlisting}[language=C]
int @!\fn{doSomething}!@(int* arr, int len) {
    for (int i = 0; i < len; ++i) {
        if (arr[i] < 4) {
            arr[i] = @!\fn{complexFunction}!@(arr[i]);
        }
    }
    for(int i = 0; i < len; ++i) {
        if (@!\fn{simpleFunction}!@(arr[i])) {
            return arr[i];
        } else {
            arr[i] = @!\fn{doSomething}!@(arr, len);
        }
    }
}
    \end{lstlisting}

\end{frame}
\begin{frame}[fragile]
    \frametitle{The Problem a Debugger Solves, Cont.}
    \begin{itemize}
        \item I want to know where the program is crashing
        \item So I do this:
    \end{itemize}
    \begin{lstlisting}[numbers=none]
#include <stdio.h>
    \end{lstlisting}
    \begin{itemize}
        \pause
        \item And this\dots
    \end{itemize}
    \begin{lstlisting}[numbers=none]
@!\fn{printf}!@("some random variable: %d", var);
    \end{lstlisting}
    \begin{itemize}
        \pause
        \item My code is full of print statements
        \item My terminal is flooded with info
    \end{itemize}


\end{frame}
\begin{frame}
    \frametitle{What I Want to Do}

    \begin{itemize}
        \item Stop my program anywhere I want
        \item See what all the variables are
        \item See how they got that way
        \item If it crashes, see what was going on when it crashed
        \item Not need to delete all those stupid print statements afterward
    \end{itemize}

\end{frame}
\begin{frame}
    \frametitle{What a Debugger Can Do for Me}

    \begin{itemize}
        \item Stop my program: breakpoints, watchpoints, on exceptions
        \item See values of all variables (and registers!)
        \item Follow \underline{call stack} back up\footnote[frame]{Underlined terms are things not everybody will know, please ask me to explain them.}
        \item Play ``what if?''
    \end{itemize}

\end{frame}
\begin{frame}
    \frametitle{Why Are We Doing This the Hard Way?}
    \begin{itemize}
        \item Builds character
        \pause
        \item Forces you to be intentional
        \item Cross-platform skills
        \item Many features hidden / not available in GUI
    \end{itemize}
    

\end{frame}
\section{Using GDB}
\begin{frame}
    \frametitle{Essential Commands}
    \framesubtitle{I forget them constantly and so will you}
    \begin{itemize}
        \item Open this: \url{http://bit.ly/GDBCheatSheet}
        \item Essential Commands:
        \begin{itemize}
            \item \code{\$ gdb EXECUTABLE} -- start debugging
            \item \code{(gdb) run ARGS\dots} -- run your program inside debugger
            \item \code{(gdb) break [FUNCTION|FILE:LINE]} -- set \underline{breakpoint}
            \item \code{(gdb) watch EXPRESSION} -- set \underline{watchpoint}
            \item \code{(gdb) step} -- \underline{single step}
            \item \code{(gdb) print EXPRESSION} -- print value
            \item \code{(gdb) kill} -- start over
            \item \code{(gdb) quit} -- exit
            \item \code{(gdb) continue} -- start running again
            \item \code{(gdb) [up|down]} -- traverse the call stack
            \item \code{(gdb) clear [FUNCTION|FILE:LINE]} -- remove watchpoint / breakpoint
        \end{itemize}
    \end{itemize}
    

\end{frame}

\begin{frame}
    \frametitle{Time for Some Real Debugging}

    \begin{itemize}
        \item Navigate to the \code{Basic\_Usage} folder
        \item Compile with \code{\$~gcc~hasfunctions.c~-lm~-o~hasfunctions}\footnote[frame]{Anybody need me to explain what the arguments are doing?}
        \item Run it with 2 int \underline{arguments}\footnote[frame]{Anybody not know what that means?}
        \pause
        \item 90\% chance you'll get a segfault
        \item Run it in gdb
    \end{itemize}

\end{frame}
\begin{frame}[fragile]
    \frametitle{Set a Breakpoint}
    \begin{lstlisting}[firstnumber=11,title={hasfunctions.c}]
int* @!\fn{doesSomethingElse}!@(double first, int count) {
    int* myArr = @!\fn{malloc}!@(count * sizeof(int));
    double trail;
    for(int i = 0; i < count; ++i) {
        myArr[i] = trail - first * 2;
        trail = myArr[i] + first * 3 - @!\fn{floor}!@(first);
    }
}
    \end{lstlisting}
    \begin{itemize}
        \item I want to stop every time the loop goes around
        \pause
        \item Answer: \code{(gdb) break hasfunctions.c:15}
        \item Breakpoints stop \textbf{before} executing the line
    \end{itemize}
\end{frame}

\begin{frame}
    \frametitle{More About Breakpoints}

    \begin{itemize}
        \item On x86 CPUs, very fast
        \item Great when you need to know what is going on at any given point in code
        \pause
        \item Not so great when you don't actually know where that point is
        \item Potential Solutions:
        \begin{itemize}
            \pause
            \item Set a shit-ton of breakpoints
            \item Use accessor / mutator functions
            \item Watchpoints
        \end{itemize}
    \end{itemize}

\end{frame}
\begin{frame}
    \frametitle{Watchpoints}
    \framesubtitle{When you have no idea when, where, or how}
    \begin{itemize}
        \item Watchpoints trigger whenever a given variable \textit{or expression}\footnote[frame]{This is most relevant if you know how to use pointers, who does?} changes
        \item Less intuitive to set than a breakpoint
        \item More technical limitations than breakpoints
        \item Can be implemented in both hardware and software, use software with\linebreak\code{(gdb) set can-use-hw-watchpoints 0}\footnote[frame]{You might need to do this if using WSL}
    \end{itemize}
\end{frame}
\begin{frame}
    \frametitle{Setting a Watchpoint}
    \begin{itemize}
        \item Open / compile / run \code{hasglobals.c}\footnote[]{Of course, watchpoints work on local vars too, but less clear to explain}
        \item Try setting a watchpoint on \code{globalvar} \pause -- \code{(gdb) watch globalVar}
        \item Try setting a watchpoint on \code{globalPtr}
        \pause
        \item What if we want to see when the memory it's pointing to changes instead? \pause --~\code{(gdb) watch *globalVar}
    \end{itemize}
\end{frame}
\begin{frame}
    \frametitle{Setting Less Intutitive Watchpoints}
    \begin{itemize}
        \item It's very obvious when you want to watch a global
        \item What about if you want to watch a local? \pause -- Breakpoint in scope, \code{(gdb)~watch~var}
        \item Try watching a value that doesn't exist yet: \code{globalPtr[2]}
        \pause
        \item GDB is fine with breakpoints that can't be read yet\footnote[frame]{The way GDB deals with situations like this reveals just how complex of a program it really is.}
        \pause
        \item However, there are limitations:
        \begin{itemize}
            \item If you change the value of globalPtr, you'll no longer be watching the same location
            \item If you manually set a watchpoint on a memory location pointed to by globalPtr, you won't follow changes to globalPtr
        \end{itemize}
    \end{itemize}

\end{frame}
\begin{frame}
    \frametitle{Explore the Call Stack}

    \begin{itemize}
        \item Go back to \code{hasfunctions.c}
        \item Break inside \code{\fn{goodNumber}}\code{()}
        \item Pretend you don't know which function called it, how would you find out? \pause -- \code{(gdb) bt}
        \pause
        \item Play around with \code{(gdb) up}
        \item You can access local scope in any one of the \underline{frames}
    \end{itemize}

\end{frame}
\begin{frame}
    \frametitle{GDB Extras}
    \begin{itemize}
        \item \code{(gdb) list *\$pc} -- Show current location in code
        \item \code{(gdb) save breakpoints FILENAME} -- Save \textit{all breakpoints}\footnote[frame]{For some reason this means breakpoints AND watchpoints.} to a file for later
        \item \code{(gdb) source FILENAME} -- Recover saved breakpoints
        \item \code{(gbd) info registers} -- See register values
    \end{itemize}
    

\end{frame}
\section{Extra Content}
\begin{frame}
    \frametitle{Dissasembly}
    \begin{itemize}
        \item Find out what your computer is \textit{really} doing
        \item Try \code{(gdb) dissasemble main}
        \pause
        \item Probably looks like garbage, try: \code{(gdb) set dissasembly-flavor intel}
        \pause
        \item Break somewhere in main, try disassembling main again \break -- You can see where the program counter is
        \pause
        \item Disassembly is useful when you don't have the source code but you need to figure out how something works.
        \item May be a topic for a future event.
    \end{itemize}
\end{frame}
\begin{frame}
    \frametitle{Syscalls}
    \begin{itemize}
        \item Requests your program makes to the OS
        \item Knowing which calls it's making and when can be very useful when dealing with files \& networking
        \item Go to \code{Syscalls}, compile and run \code{makesSyscalls.cpp}
        \pause
        \item Wow, look at all that gibberish!
        \item Pay attention to \code{openat, read, write}
        \item Notice how there's only one read call, why? \pause -- Compiler and/or OS optimization
    \end{itemize}
\end{frame}
\begin{frame}
    \frametitle{Debugging in VsCode}

    \begin{itemize}
        \item On linux, VsCode's debugger actually uses GDB behind the scenes
        \item Watch as I set it up and use it
    \end{itemize}
\end{frame}
\section{}
\begin{frame}
    \frametitle{Thanks for Coming}
    \begin{itemize}
        \item Next Week: Research Meeting
    \end{itemize}
\end{frame}
\end{document}